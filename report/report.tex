\documentclass[10pt, conference, letterpaper]{IEEEtran}

\usepackage{algorithm}
\usepackage{algorithmicx}
\usepackage{algpseudocode}
\usepackage{amsfonts}
\usepackage{amsmath}
\usepackage{amssymb}
\usepackage[ansinew]{inputenc} 
\usepackage{xcolor}
\usepackage{mathtools}
\usepackage{graphicx}
\usepackage{caption}
\usepackage{subcaption}
\usepackage{import}
\usepackage{multirow}
\usepackage{cite}
\usepackage[export]{adjustbox}
\usepackage{breqn}
\usepackage{mathrsfs}
\usepackage{acronym}
%\usepackage[keeplastbox]{flushend}
\usepackage{setspace}
\usepackage{bm}
\usepackage{stackengine}

\usepackage{listings}

\lstset{%
 backgroundcolor=\color[gray]{.85},
 basicstyle=\small\ttfamily,
 breaklines = true,
 keywordstyle=\color{red!75},
 columns=fullflexible,
}%

\lstdefinelanguage{BibTeX}
  {keywords={%
      @article,@book,@collectedbook,@conference,@electronic,@ieeetranbstctl,%
      @inbook,@incollectedbook,@incollection,@injournal,@inproceedings,%
      @manual,@mastersthesis,@misc,@patent,@periodical,@phdthesis,@preamble,%
      @proceedings,@standard,@string,@techreport,@unpublished%
      },
   comment=[l][\itshape]{@comment},
   sensitive=false,
  }

\usepackage{listings}

% listings settings from classicthesis package by
% Andr\'{e} Miede
\lstset{language=[LaTeX]Tex,%C++,
    keywordstyle=\color{RoyalBlue},%\bfseries,
    basicstyle=\small\ttfamily,
    %identifierstyle=\color{NavyBlue},
    commentstyle=\color{Green}\ttfamily,
    stringstyle=\rmfamily,
    numbers=none,%left,%
    numberstyle=\scriptsize,%\tiny
    stepnumber=5,
    numbersep=8pt,
    showstringspaces=false,
    breaklines=true,
    frameround=ftff,
    frame=single
    %frame=L
}

\renewcommand{\thetable}{\arabic{table}}
\renewcommand{\thesubtable}{\alph{subtable}}

\DeclareMathOperator*{\argmin}{arg\,min}
\DeclareMathOperator*{\argmax}{arg\,max}

\def\delequal{\mathrel{\ensurestackMath{\stackon[1pt]{=}{\scriptscriptstyle\Delta}}}}

\graphicspath{{./figures/}}
\setlength{\belowcaptionskip}{0mm}
\setlength{\textfloatsep}{8pt}

\newcommand{\eq}[1]{Eq.~\eqref{#1}}
\newcommand{\fig}[1]{Fig.~\ref{#1}}
\newcommand{\tab}[1]{Tab.~\ref{#1}}
\newcommand{\secref}[1]{Section~\ref{#1}}

\newcommand\MR[1]{\textcolor{blue}{#1}}
\newcommand\red[1]{\textcolor{red}{#1}}
\newcommand{\mytexttilde}{{\raise.17ex\hbox{$\scriptstyle\mathtt{\sim}$}}}

%\renewcommand{\baselinestretch}{0.98}
% \renewcommand{\bottomfraction}{0.8}
% \setlength{\abovecaptionskip}{0pt}
\setlength{\columnsep}{0.2in}

% \IEEEoverridecommandlockouts\IEEEpubid{\makebox[\columnwidth]{PUT COPYRIGHT NOTICE HERE \hfill} \hspace{\columnsep}\makebox[\columnwidth]{ }} 

\title{Small footprint deep neural network architectures for command speech recognition}

\author{María Emilia Charnelli$^\dag$, Lorenti Luciano Rolando$^\dag$
\thanks{$^\dag$Department of Information Engineering, University of Padova, email: \{mariaemilia.charnelli, lucianorolando.lorenti\}@studenti.unipd.it}

} 

\IEEEoverridecommandlockouts

\newcounter{remark}[section]
\newenvironment{remark}[1][]{\refstepcounter{remark}\par\medskip
   \textbf{Remark~\thesection.\theremark. #1} \rmfamily}{\medskip}

\begin{document}

\maketitle

\begin{abstract}

Keyword  spotting  (KWS)  constitutes  a  major  component  of human-technology interfaces. KWS systems enable hands-free speech recognition experience by detecting a trigger phrase used to initiate the interaction with a device. The ubiquity of mobile devices has promoted research of models capable of doing KWS with small memory footprint and low computational power requirements.
In recent years, machine learning techniques, such as deep neural networks, have proven to be useful for keyword spotting. In particular, small footprint models has been proposed using different techniques for  
Residual neural networks (ResNet) and Vision Transformers  (ViT) are recent deep neural network architectures that showed promising results in a broad range of areas. In this context, a comparison between small footpring residual neural network models and ViT was performed. We  analyze  the  effect  of architecture  parameters and the number of trainable parameters of the model. 
The resulting analysis show that ResNet, even with a reduced number of weights can obtain good results with small . while ViT transformer require a bit more carefulyl at the moment of choosing the architeture.
\end{abstract}

\IEEEkeywords
Keyword  spotting, Neural Networks, Vision Transformers, Residual Neural Networks, Recurrent Neural Networks. 
\endIEEEkeywords


% !TEX root = template.tex

\section{Introduction}
\label{sec:introduction}

\red{Maximum length for the whole report is 9 pages. Abstract, introduction and related work should take max two pages.}\\


Automatic Speech Recognition (ASR) is designed to automatically transcribe the human voice into text. Human-computer interfaces (HCI) like Google Assistant, Microsoft Cortana, Amazon Alexa, Apple Siri and others rely on speech recognition to execute commands dictated by voice.   In the past years, due to the reduced computational power of the mobile devices, the architecture of the application consisted on transmit audio the audio captured from the device to a more powerful server, while the decoded results are transmitted to the user. Coupled with the tremendous growth and adoptionof smartphones, tablets and other consumer devices, these improvements have resulted in speech becoming one of the primary modes of interaction with such devices. Replacing such a server-based system with one that can run entirely on-device has important implications from a reliability, latency,and privacy perspective, and has become an active area of research.Therefore, industry applications which do not have the benefit ofuninterrupted broadband internet connection cannot incorporate speech recognition in thesame manner.  The development of lightweight speech command models would enable thedevelopment of a multitude of novel engineering applications, such as voice-controlled robotsfor critical missions and assistive devices that can operate in areas without internet coverage..
Prominent examples includewakeword detection(i.e., recognizingspecific words or phrases) [4,5,6,7], as well as large vocabularycontinuous speech recognition (LVCSR) [8, 9]

In the past years, deep learning (Yu \& Deng, 2014) has beensuccessfully applied in ASR to boost the recognition ac-curacy.  Very recently, CNN becomes an attractive modelin ASR, which transforms speech signals into feature mapsas used in computer vision (LeCun \& Bengio, 1998). Com-pared to other deep learning architectures, CNN has severaladvantages:  1) CNN is suited to exploit local correlationsof  human  speech  signals  in  both  time  and  frequency  di-mensions. 2) CNN has the capacity to exploit translationalinvariance in signals.






\noindent \textbf{Recommended structure for the intro:} you may use the following structure. 
\begin{itemize}
\item \textbf{General (short) intro:} One paragraph to introduce your work, describing the scenario {\it at large}, its relevance, to prepare the reader to what follows and convince her/him that the paper focuses on an important setup/problem. Please keep this part short (I usually do five to eight lines), as this part is rather standard, \textbf{but} at the same time it has to be there. 
\item \textbf{Put the problem into perspective:} A second paragraph where you immediately delve into the specific problem that you are tackling, starting to detail your contribution. Here, you describe the importance of such problem, providing examples (citing papers from the literature, possibly recent ones) of previous solutions attempts, and of why these failed {\it to provide a complete answer}. This second paragraph should not be too long, as otherwise the reader will get bored and will abandon your paper... It should be concisely written, something like 5 to 10 lines.
\item \textbf{Present the paper contribution:} A third paragraph were you state what you do in the paper, this should also be concisely written. A good rule of thumb is to make it max 10/15 lines. Here, you should state up front:
\begin{enumerate}
\item \textbf{problem}: the problem at stake, 
\item \textbf{relevance}: the relevance and timeliness of what you propose, 
\item \textbf{approach}: the technique/approach you use, possibly underlying its novelty, efficiency, 
\item \textbf{value}: underline the value/novelty of your proposal referencing (recent) papers from the literature,
\item \textbf{applicability}: tell the reader how she/he can take advantage of your work, e.g., how your work/results can be reused/exploited to achieve further scientific, technical or practical (integrated into products?) goals.
\end{enumerate}
\item \textbf{Summary of contributions:} After this, you may want to provide an itemized list to summarize the paper contributions. Rule of thumb: from three to six items, from three to four lines each.
\item \textbf{Closing (paper structure):} You finish up by detailing the paper structure, this should be three to four lines. It is customary to do so, although I admit it may be of little use. It usually goes like: {\it ``This report (paper) is structured as follows. In Section II we describe the state of the art, the system and data models are respectively presented in Sections~III and~IV. The proposed signal processing technique is detailed in Section~V and its performance evaluation is carried out in Section~VI. Concluding remarks are provided in Section~VII.''}
\end{itemize}

\begin{remark} 
Lately, I tend to write introduction plus abstract within a single page. This forces me to focus on the important messages that I want to deliver about the paper, leaving out all the ``blah blah''. \textbf{Remember:} 1) {\it less is more}, 2) writing a compact ({\it snappy}) piece of technical text is much more difficult than writing lengthy stuff with no space constraints.
\end{remark}


% !TEX root = template.tex

\section{Related Work}

Deep learning models have shown state-of-the-art performance in speech recognition \cite{hinton2012deep}, \cite{sainath2013deep}. Following the work in [28], DBNs are trained in layer-wise fashion followed by end-to-end fine-tuning for speech applications as shown in Fig. 2 above. The DBN architecture and training process have been extensively tested on several large-vocabulary speech recognition datasets including TIMIT, Bing-Voice-Search speech, Switchboard speech, Google Voice Input speech, YouTube speech, and the English-Broadcast-News speech data set. DBNs significantly outperform state-of-the-art methods in speech recognition when compared to highly tuned Gaussian mixture model (GMM)Hidden Markov Model (HMM). SAEs likewise are shown to outperform GMM-HMM on Cantonese and other speech recognition tasks [43].


The success of future speech and vision systems depends on accessibility and adaptability to a variety of platforms that eventually drive the prospect of commercialization. While some platforms are intended for public and personal use, there are other commercial, industrial, and online-based platforms - all of which require seamless integration of IS. However, state-of-the-art deep learning models have challenges in adapting to embedded hardware due to large memory footprint, high computational complexity, and high-power consumption. This has driven research on improving system performance of compact architectures in resource restricted platforms. The following sections highlight some of the major research efforts in integrating sophisticated algorithms in resource restricted user platforms.
4.1. Speech recognition on mobile platforms


The authors in \cite{chen2014small} have proposed a low-latency keyword detection method for mobile users using a deep learning-based technique and termed it as ‘deep KWS’. The deep KWS method has not only been proven suitable for low-powered embedded systems but also has outperformed the baseline Hidden Markov Models for both noisy and noise-free audio data. The deep KWS uses a fully connected DNN with transfer learning  based on speech recognition. The network is further optimized for KWS with end-to-end fine-tuning using stochastic gradient descent. Sainath et al. \cite{sainath2015convolutional} have introduced a similarly small footprint KWS system based on CNNs. Their proposed CNN uses fewer parameters than a standard DNN model, which makes the proposed system more attractive for platforms with resource constraints. 

In \cite{tang2018deep}

Similar to KWS systems, automatic speech recognition (ASR) [170] has become increasingly popular with mobile devices as it alleviates the need for tedious typing on small mobile devices. Google provides ASR-based search services [166] on Android, iOS, and Chrome platforms. Apple iOS devices are equipped with a conversational assistant named Siri. Mobile users can also type texts or emails by speech on both Android and iOS devices [171]. However, ASR service is contingent on the availability of cellular mobile network since the recognition task is performed on a remote server. This is a limitation since mobile network strength can be low, intermittent, or even absent at places. Therefore, developing an accurate speech recognition system in real-time, embedded on standalone modern mobile devices, is still an active area of research.




LSTM
Chen et al. [169] in another study propose the use of LSTM for the KWS task. The inherent recurrent connections in LSTM can make the KWS task suitable for resource restricted platforms by improving computational efficiency. To support this, the authors further show that the proposed LSTM outperforms a typical DNN-based KWS method. A typical framework for deep learning based KWS system is shown in Fig. 5.



% !TEX root = template.tex

\section{Processing Pipeline}
\label{sec:processing_architecture}
\begin{figure}
\caption{Schematic}
\label{system_scheme}
\end{figure}

In a general way, the system used consists on the main components. An scheme of the system is shown on figure \ref{system_scheme}.  CNN is an image classification technique and one of the major challenges in speech and acoustic event recognition has been how to best represent the audio signal using an image for this purpose. Two common approaches have been seen in addressing this problem. Firstly, the audio signal is converted to spectrogram images \cite{zhang2015robust}. Secondly, a mel-filter, as used in computing mel-frequency cepstral coefficients (MFCC), is used to form an image-like representation \cite{abdel2014convolutional]}.  To ensure that all images are of an equal size, the audio signal is divided into a fixed number of frames \cite{We refer this as the mel-spectrogram.}. The first component consist on a feature extraction module  from the audio signals. The second module consists on the convolutional neural network architecture. This model will output a probability of each segment to belong to a class. Finally, a post-processing module, takes for each audio input the list of probabiltiies, and produces the final predicted class for that audio.

 



CNNs run a small window over the input image atboth training and testing time, so that the weights of the networkthat looks through this window canlearn from various featuresof the input data regardless of their absolute position within theinput.Weight sharing, or to be more precise in our present situ-ation,full weight sharingrefers to the decision to use the sameweights at every positioning of the window. CNNs are also often said to be local because the individual units that are computed at a particular positioning ofthe window depend upon featuresof the local region of the image that the window currently looksupon.
 Time presents no immediate problem from the standpoint of locality. Likeother DNNs for speech, a single window of input to the CNNwill consist of a wide amount of context (9–15 frames). Asfor frequency, the conventional use of MFCCs does present amajor problem because the discrete cosine transform projectsthe spectral energies into a new basis that may not maintain lo-cality. In this paper, we shall use the log-energy computed di-rectly from the mel-frequency spectral coefficients (i.e., with noDCT), which we will denote asMFSC features. These will beused to represent each speech frame in order to describethe acoustic energy distributionin each of several different frequency bands

There exist several different alternatives to organizing theseMFSC features into maps for the CNN. First, as shown inFig. 1(b), they can be arranged as three 2-D feature maps,each of which represents MFSC features (static, delta anddelta-delta) distributed along both frequency (using the fre-quency band index) and time (using the frame number withineach context window). In this case, a two-dimensional con-volution is performed (explained below) to normalize bothfrequency and temporal variations simultaneously. 


We evaluated our models using Google’s Speech CommandsDataset  [9],  which  was  released  in  August  2017  under  aCreative  Commons  license.2The  dataset  contains  65,000one-second long utterances of 30 short words by thousands ofdifferent people, as well as background noise samples such aspink noise, white noise, and human-made sounds.  The blogpost  announcing  the  data  release  also  references  Google’sTensorFlow implementation of Sainath and Parada’s models,which provide the basis of our comparisons


The output of each must be post-processed

\section{Signals and Features}
The audio signals For  feature  extraction,   we  first  apply  a  band-pass  filterof  20Hz/4kHz  to  the  input  audio  to  reduce  noise.    For the speech regions, we generate acoustic features based on 40-dimensional log-filterbank energies computed every 10 ms over a window of 25 ms. 
    

Contiguous frames are stacked to add sufficient left and right context. The input window is asymmetric since each additional frame of future context adds 10 ms of latency to the system. For our Deep KWS system, we use 10 future frames and 30frames in the past.


Frames with size $40x40$

No frames the every data has a a shape of 100x40
NFFT=256, nfilt=40
numcep=40, nfilt=nfilt, nfft=51

\section{Learning Framework}
 
 Basically we used two different approaches, one based on a Residual Neural Network (ResNet) proposed by Sainath \cite{sainath2013deep}, \cite{tang2018deep} and the other based on a Vision Transformer (ViT) model proposed by Alexey Dosovitskiy et al. [cite].


\subsection{Residual layers}
In particular, we explore the use of residual learning techniques anddilated convolutions.


\begin{table}
	\centering
	\begin{tabular}{|l|l|l|l|}
    \hline
    name & residuals  & filters & total parameters \\
    \hline
    res3 & 3 & 19 & 20,380 \\
    \hline
    res6 & 6 & 19 & 40,330 \\
    \hline
    res9 & 9 & 19 & 60,280 \\
    \hline
    \end{tabular} 
\label{table:results}
\caption{Parameters of the ResNet models}
\end{table}

\subsection{Transformer layers}

Dosovitskiy et al. introduced the Vision Transformer (ViT) [2] and showed that Transformers can learn high-level image features by computing self-attention between different image patches. 	

Transformers were proposed by Vaswani et al. (2017) for machine translation, and have since become the state of the art method in many NLP tasks.
	 	 	 		
Alexey Dosovitskiy et al. created the ViT model for image classification using the Transformer architecture with self-attention to sequences of image patches, without using convolution layers.
	
 To apply this approach in our scenario, we split an audio signal into fixed\-size patches, linearly embed each of them, add position embeddings, and feed the resulting sequence of vectors to a standard Transformer encoder. In order to perform classification, we use the standard approach of adding an extra learnable "classification token" to the sequence.


An investigation into the application of the Transformer architecture to keyword spotting, finding that applying self-attention is more effective in the time domain than in the frequency domain.

\begin{table}
	\centering
	\begin{tabular}{|l|l|l|l|}
    \hline
    name & patches  & projection dim &  total parameters \\
    \hline
    vit5x4 & 5x4 & 20 & 20xxx \\
    \hline
    vit10x8 & 10x8 & 20 & 20xxx \\
    \hline
    vit20x10 & 20x10 & 20 & 20xxx \\
    \hline
    \end{tabular} 
\label{table:results}
\caption{Parameters of the Transformer models}
\end{table}


\subsection{Experimental Setup}

The Speech Commands Dataset was
split into training, validation, and test sets, with 80\% training,
10\% validation, and 10\% test.

To generate training data, we followed Google's preprocessing
procedure by adding background noise to each
sample with a probability of 0:8. The noise is chosen randomly from the background noises provided
in the dataset. After evaluated different ratios of noise to add an every signal, we defined a ratio of 0.4 in order that a human ear can still listening the recorded commands with some background noise. 

For evaluate the models we use the accuracy metric and also we compute the total time used for training the models.

\subsection{Model Training}

We used stochastic gradient descent with a momentum of x and a starting learning rate of 0.01 for the ResNet model and 0.001 for the Transformer model, which is multiplied by x on plateaus. 

We used a mini-batch size of 64 and a total of 27 epochs to train our models.

Also, we add Early Stopping with patience 5 and for Plateau patience 2, and a min learning rate of 0.00001.

% !TEX root = template.tex

\section{Results}
\label{sec:results}

%In this section, you should provide the numerical results. You are free to decide the structure of this section. As a general ``rule of thumb'', use plots to describe your results, showing, e.g., precision, recall and \mbox{F-measure} as a function of the system (learning) parameters. You can also show the precision matrix. 


The results obtained with these models are shown in Table \ref{table:results}, which also shows the 95\% confidence intervals from tree  different trials.

\begin{table}
	\centering
	\begin{tabular}{|l|l|l|l|}
    \hline
    model  & parameters & accuracy  & time \\
    \hline
    res3 & 20,380 & 0.8445 & 5849ms \\
    \hline
    res6 & 40,330 & 0.8560 & 7384ms\\
    \hline
    res9 & 60,280 & 0.8629 & 9814ms \\
    \hline
    vit10x10 & 42,756 & 0 & 0  \\
    \hline
    vit10x8  & 54,044 & 0 & 0  \\
    \hline
    vit20x10  & 26,284 & 0.7597 & 3242ms \\
    \hline
    \end{tabular} 
\label{table:results}
\caption{Results obtained per each model over 3 trials}
\end{table}

\begin{figure}[h]
    \centering
    \includegraphics[width=0.5\textwidth]{confusion_matrix_res3_keyword.png}
    \label{fig:confusionmatrix}
    \caption{Confusion Matrix of the model res3}
\end{figure}


Checking the confusion matrix of model res3 \ref{fig:confusionmatrix}, we can observe well the accuracy is no so high. We can notice that the best label that the model predicts is unknown, and also we can observe that the model sometimes doesn't predict well for labels off and up.  

%
%\begin{remark}
%Present the material in a progressive and logical manner, starting with simple things and adding details and explaining more complex findings as you go. Also, do not try to explain/show multiple concepts within the same sentence. Try to \textbf{address one concept at a time}, explain it properly, and only then move on to the next one.
%\end{remark}
%
%\begin{remark}
%The best results are obtained by generating the graphs using a vector type file, commonly, either \texttt{encapsulated postscript (eps)} or \texttt{pdf} formats. To plot your figures, use the Latex \texttt{\textbackslash includegraphics} command. Lately, I tend to use pdf more.
%\end{remark}
%
%\begin{remark}
%If your model has hyper-parameters, show selected results for several values of these. Usually, tables are a good approach to concisely visualize the performance as hyper-parameters change. It is also good to show the results for different flavors of the learning architecture, i.e., how architectural choices affect the overall performance. An example is the use of CNN only or CNN+RNN, or using inception for CNNs, dropout for better generalization or attention models. So you may obtain different models that solve the same problem, e.g., CNN, CNN+RNN, CNN+inception, etc.
%\end{remark}


% !TEX root = template.tex

\section{Concluding Remarks}
\label{sec:conclusions}

In this project we put the emphasis in create simple models, with a low number of parameters and that don't need to much time for train it.
Comparing our results with Sainath, Tang we obtained a lower performance, but they create models with hundreds of parameters and also in Tang they only train the models with the small dataset of 65000 instead of 350000.
Therefore, we obtained more simple models with less accuracy but they still competitive in the performance and are very fast to ran and to use in applications.

Besides our expectations and the last research papers about Transformers applied in Computer Vision, our best models where using residuals layer in the models. We believe that the model using Transformer need more tests with different hyper-parameters or only more epochs to see a more competitive result. These kind of models are very fast to train and its motivated to use more epochs to give more time for the model to learn.

\red{This section should take max half a page, I personally find it difficult to come up with really useful observations, I mean ones that bring a new contribution with respect to what you have already expounded in the ``Results'' section. In case you have some serious stuff to write, you may also extend the section to 3/4 of a page :-).}\\

In many papers, here you find a summary of what done. It is basically an abstract where instead of using the present tense you use the past participle, as you refer to something that you have already developed in the previous sections. While I did it myself in the past, I now find it rather useless.\\ 

\MR{\textbf{What I would like to see here is:} 
\begin{enumerate}
\item a very short summary of what done, 
\item some (possibly) intelligent observations on the relevance and {\it applicability} of your algorithms / findings, 
\item what is still missing, and can be added in the future to extend your work.\\
\end{enumerate}
The idea is that this section should be {\it useful} and not just a repetition of the abstract (just \mbox{re-phrased} and written using a different tense...).}\\

\red{\textbf{Moreover:} being a project report, I would also like to see a specific paragraph stating 
\begin{enumerate}
\item[4)] what you have learned, and 
\item[5)] any difficulties you may have encountered.
\end{enumerate}}


\section{Exam rules}

What you need to do to pass the exam:
\begin{itemize}
\item Optional: team up with another student. Max. group size is \textbf{two students} per group;
\item Identify a project to work on, devise your own neural network architecture and test it on the provided dataset;
\item \textbf{Prepare a written project report} including: i) diagrams, ii) configuration pars, iii) results, iv) your discussion;
\item \textbf{Prior to presenting your work}: upload i) your written report and ii) the code;
\item \textbf{Present your work} using slides (max. duration is 20 minutes): take turns in presenting your work, your individual contribution to the project should clearly emerge. Optional: a final and quick demo with running code is appreciated and will be considered in the calculation of the final grade (see below). 
\end{itemize}

Your final grade will be obtained taking into account the following criteria:
\begin{itemize} 
\item \textbf{Project} (60 points): originality (10 pt.) - data preprocessing techniques (10 pt.) - learning architectures (20 pt.) - comparison against other/existing approaches (10 pt.) - live demo of the code (10 pt.)
\item \textbf{Written report} (40 points): clarity of exposition (10 pt.) - completeness (10 pt.) - analysis of results (number and type of metrics used) (20 pt.)
\item \textbf{Oral exposition} (20 points): duration (your talk must take max. 20 minutes, using slides) (10 pt.) - clarity of exposition (10 pt.)
\end{itemize}

The final grade will be computed as
\begin{equation}
\textrm{grade} = \frac{\textrm{tot\_points} \times 30}{110}
\end{equation}

\bibliography{biblio}
\bibliographystyle{ieeetr}

\end{document}


