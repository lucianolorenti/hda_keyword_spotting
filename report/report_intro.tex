% !TEX root = template.tex

\section{Introduction}
\label{sec:introduction}

\red{Maximum length for the whole report is 9 pages. Abstract, introduction and related work should take max two pages.}\\


Automatic Speech Recognition (ASR) is designed to automatically transcribe the human voice into text. Human-computer interfaces (HCI) like Google Assistant, Microsoft Cortana, Amazon Alexa, Apple Siri and others rely on speech recognition to execute commands dictated by voice.   In the past years, due to the reduced computational power of the mobile devices, the architecture of the application consisted on transmit audio the audio captured from the device to a more powerful server, while the decoded results are transmitted to the user. Coupled with the tremendous growth and adoptionof smartphones, tablets and other consumer devices, these improvements have resulted in speech becoming one of the primary modes of interaction with such devices. Replacing such a server-based system with one that can run entirely on-device has important implications from a reliability, latency,and privacy perspective, and has become an active area of research.Therefore, industry applications which do not have the benefit ofuninterrupted broadband internet connection cannot incorporate speech recognition in thesame manner.  The development of lightweight speech command models would enable thedevelopment of a multitude of novel engineering applications, such as voice-controlled robotsfor critical missions and assistive devices that can operate in areas without internet coverage..
Prominent examples includewakeword detection(i.e., recognizingspecific words or phrases) [4,5,6,7], as well as large vocabularycontinuous speech recognition (LVCSR) [8, 9]

In the past years, deep learning (Yu \& Deng, 2014) has beensuccessfully applied in ASR to boost the recognition ac-curacy.  Very recently, CNN becomes an attractive modelin ASR, which transforms speech signals into feature mapsas used in computer vision (LeCun \& Bengio, 1998). Com-pared to other deep learning architectures, CNN has severaladvantages:  1) CNN is suited to exploit local correlationsof  human  speech  signals  in  both  time  and  frequency  di-mensions. 2) CNN has the capacity to exploit translationalinvariance in signals.






\noindent \textbf{Recommended structure for the intro:} you may use the following structure. 
\begin{itemize}
\item \textbf{General (short) intro:} One paragraph to introduce your work, describing the scenario {\it at large}, its relevance, to prepare the reader to what follows and convince her/him that the paper focuses on an important setup/problem. Please keep this part short (I usually do five to eight lines), as this part is rather standard, \textbf{but} at the same time it has to be there. 
\item \textbf{Put the problem into perspective:} A second paragraph where you immediately delve into the specific problem that you are tackling, starting to detail your contribution. Here, you describe the importance of such problem, providing examples (citing papers from the literature, possibly recent ones) of previous solutions attempts, and of why these failed {\it to provide a complete answer}. This second paragraph should not be too long, as otherwise the reader will get bored and will abandon your paper... It should be concisely written, something like 5 to 10 lines.
\item \textbf{Present the paper contribution:} A third paragraph were you state what you do in the paper, this should also be concisely written. A good rule of thumb is to make it max 10/15 lines. Here, you should state up front:
\begin{enumerate}
\item \textbf{problem}: the problem at stake, 
\item \textbf{relevance}: the relevance and timeliness of what you propose, 
\item \textbf{approach}: the technique/approach you use, possibly underlying its novelty, efficiency, 
\item \textbf{value}: underline the value/novelty of your proposal referencing (recent) papers from the literature,
\item \textbf{applicability}: tell the reader how she/he can take advantage of your work, e.g., how your work/results can be reused/exploited to achieve further scientific, technical or practical (integrated into products?) goals.
\end{enumerate}
\item \textbf{Summary of contributions:} After this, you may want to provide an itemized list to summarize the paper contributions. Rule of thumb: from three to six items, from three to four lines each.
\item \textbf{Closing (paper structure):} You finish up by detailing the paper structure, this should be three to four lines. It is customary to do so, although I admit it may be of little use. It usually goes like: {\it ``This report (paper) is structured as follows. In Section II we describe the state of the art, the system and data models are respectively presented in Sections~III and~IV. The proposed signal processing technique is detailed in Section~V and its performance evaluation is carried out in Section~VI. Concluding remarks are provided in Section~VII.''}
\end{itemize}

\begin{remark} 
Lately, I tend to write introduction plus abstract within a single page. This forces me to focus on the important messages that I want to deliver about the paper, leaving out all the ``blah blah''. \textbf{Remember:} 1) {\it less is more}, 2) writing a compact ({\it snappy}) piece of technical text is much more difficult than writing lengthy stuff with no space constraints.
\end{remark}
